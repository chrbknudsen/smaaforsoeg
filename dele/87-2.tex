%\includepdf[pages=-]{pdfs/.pdf}
\emne{Småforsøg med protolytiske reaktioner i vandige saltopløsninger}

\danskkemi{87 2 18 }

\deloverskrift{Saltes reaktion med vand.}


De anførte forsøg er en del af en samling, der blev udviklet under ledelse af R.W. Asmussen på Kemisk Laboratorium B på DTU i 1950'erne

Den protolytiske tilstand i opløsninger af salte er bestemt af de tilstedeværende ioners protolytiske karakter.

Ved brug af protolytiske indikatorer kan man danne sig et skøn over vandige saltopløsningers pH-værdi. De protolytiske indikatorer skifter farve i et for hver indikator karakteristisk område.


\deloverskrift{Forberedelse}

Ved forsøgene får man brug for

indikatorer: methylorange $\cdotp$ bromthymolblåt $\cdotp$ phenolphthalein

salte: jernalun $\cdotp$ aluminiumnitrat $\cdotp$ natriumchlorid
       natriumhydrogencarbonat $\cdotp$ natriumcarbonat
                                natriumsulfid



\deloverskrift{Fremgangsmåde}

Følgende opløsninger fremstilles:

1. 4 reagensglas fyldes halvt med vand, og 4-5 dråber methylorange tilsættes. Til hvert enkelt glas sættes hhv. én spatelfuld 1) jernalun, 2) aluminiumnitrat, 3) natriumhydrogencarbonat og 4) natriumchlorid.

2. Til 4 reagensglas med vand og 4-5 dråber bromthymolblåt sættes hhv. én spatelfuld 1) aluminiumnitrat, 2) natriumchlorid, 3) natriumhydrogencarbonat og 4) natriumcarbonat.

3. Til 4 reagensglas med vand og 4-5 dråber phenolphthalein sætets hhv. en spatelfuld 1) natriumchlorid, 2) natriumhydrogencarbonat, 3) natriumcarbonat og 4) natriumsulfid.



Litteratur

R.W. Asmussen m.fl. 1955: Vejledning til øvelser i Kemi for M, B og E. Polyteknisk Forening: 47.
