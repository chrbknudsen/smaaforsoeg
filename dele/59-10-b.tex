\pdf{pdfs/1978-59-10-358.pdf}

\emne{Springvandsforsøget på en ny måde}
\danskkemi{}
\forfatter{Peter Norrild}

Selvom HCl-Springvandsforsøget er en elegant måde at vise hydrogenchlorids
vandopløselighed på, er det muligt at udføre et forsøg over samme tema,
der er endnu bedre og endnu mere instruktivt.

A. Fremstilling af HCl
Fyld et reagensglas 2/3 op med en blanding af 6 g fintpulveriseret NaCl og
12 g vandfrit NaHSO4. Sæt en prop med hul i reagensglasset og forbind
herefter en urinpose med reagensglasset. Ophed blandingen med en bunsenbrænder.
På et par minutter fyldes urinposen med ca. 2,5 l HCl gas. Luk posen med en
slangeklemme.

B. HCl + H2O?
Forbind en 100 ml tragt med urinposens slange. Hæld vand i tragten og lad
posen hænge ned. Luk kortvarigt op for klemhanen, så halvdelen af vandet i
tragten løber ned i posen. Det siger svup! Posen skrumper ind, og i bunden
har man ca. 40 ml 60-65 gradtegn varm saltsyre.

ad A. Natriumhydrogensulfat forhandles normalt som det vandholdige NaHSO4 prik H2O,
der jo bedre kendes som toiletpulver. Dette kan ikke bruges til forsøget på
grund af vandindholdet. Man kan hos Struers købe vandfrit NaHSO4 (Merck 6351)
for en beskeden merpris. Med den beskrevne metode elimineres risikomomenter, der
normalt er forbundet med fremstilling af HCl ud fra NaCl og konc. H2SO4.
Reaktionen er
NaHSO4 + NaCl -> Na2SO4 + HCl.
Naturligvis skal man ikke glemme, at det faste NaHSO4 også er en stærk syre.

ad B. Ved Springvandsforsøget bemærker man ikke varmetoningen ved reaktionen
H2O + HCl -> H3O+ + Cl-
på grund af den store vandmængde der suges op i kolben. Ved urinpose-forsøget
er det muligt at konstatere en kraftig varmetoning og heraf slutte, at H2O reagerer
med HCl. Det analoge forsøg med NH3 og H2O giver en noget mindre varmetoning.

HCl + H2O -> H3O+ + Cl-
DELTAH0 = -75,0 kJ prik mol-1

NH3 (g) -> NH3 (aq)
(NH3 + H2O <-> NH4+ + OH-)   krøllet parantes

DELTAH0 = -34,6 kJ prik mol-1

Hældes de 40 g vand i 2,5 l HCl = 1/10 mol, udvikles der 7,50 kJ

Temperaturstigningen DELTAt findes af ligningen
40 prik DELTAt prik 4,19 = 7500 ; DELTAt = 44,7

Er vandets temperatur før forsøgets start 20 Gradtegn C, bliver denne teoretisk
64,7 gradtegn C. Den fremstillede saltsyre bliver altså mere end håndvarm, og
PVC-urinposen får en noget slatten karakter. Hvis man erstatter tragten med en
plastsprøjte og tilsætter en nøje afmålt mængde vand (f.eks. 40 ml), findes
varmetoningen at være i god overensstemmelse med den beregnede.
