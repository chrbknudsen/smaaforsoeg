\pdf{pdfs/1978-59-10-358.pdf}

\emne{Kemisk Ligevægt}
\danskkemi{}
\forfatter{Ole Bostrup}

En kemisk ligevægt kan demonstreres ved reaktionen mellem jern(III)-ioner
og thiocyanat-ioner
Fe3+ (aq) + SCN- (aq) <-> Fe (SCN)2+ (aq)

Hvor der dannes et rødfarvet komplex FeSCN2+.
Til forsøget skal bruges 1M NH4SCN, 0,1M FeCl3 og 0,1 M AgNO3 samt fast
Na2HPO4.

a. Til 100 cm3 vand sættes 10 dråber 1M NH4SCN. Tilsæt 10 dråber 0,1M FeCl3.
Bemærk Fe(SCN)2+ farven efter omrøren.
b. Placer 5 reagensglas i et stativ og fyld dem halvt med opløsningen fra bægerglasset:
1. Bruges til sammenligning
2. Tilsæt nogle dråber 1M NH4SCN
3. Tilsæt nogle dråber 0,1 M FeCl3
4. Tilsæt nogle dråber 0,1 M AgNO3
5. Tilsæt en spatelfuld fast Na2HPO4
