\pdf{pdfs/1979-60-1-34.pdf}

\emne{Forsøg der belyser opløselighedsproduktet}
\danskkemi{}
\forfatter{Niels Berg}

1. Blyjodid, \ch{PbI2}, er velegnet til at demonstrere virkningen af fælles ioner
på opløseligheden.
Fremstilling af blyjodid: Opløs 1,0 g Pb(NO3)2 i 300 ml kogende vand og tilsæt
en opløsning af 4,0 g kaliumjodid i lidt vand. Udfældes der herved blyjodid,
varmes indtil alt er gået i opløsning. Af den farveløse opløsning udskilles
det gule blyjodid ved afkøling.
Blyjodid suges fra på glasfilter og vaskes et par gange med vand. Noget af
det endnu våde PbI2 kommes i en medicinflaske (500 ml), og efter tilsætning af
vand lukkes flasken med prop og rystes i 10 min.

I hvert af 2 cylinderglas kommes 50 ml af den (filtrerede) mættede opløsning
af PbI2.
Til det ene cylinderglas sættes lidt KI-opløsning. Der udfældes straks PbI2.
Til det andet cylinderglas sættes 1 ml mættet blynitrat-opløsning. Der Udfældes
langsomt større krystaller af PbI2.

Anm: 50 ml \ch{PbI2} + 1/2 ml \ch{Pb(NO3)}, giver fældning.
50 ml \ch{PbI2} + 1 ml \ch{Pb(NO3)}, giver fældning
50 ml \ch{PbI2} + 2 ml \ch{Pb(NO3)}, giver fældning
50 ml \ch{PbI2} + 4 ml \ch{Pb(NO3)}, giver ikke fældning

2. Sølvacetat, \ch{CH3COOAg}, kan bruges til et analogt forsøg:
0,50 g sølvnitrat og 0,40 g krystallinsk natriumacetat, CH3COONa,3H2O *) opløses
hver for sig i 25 ml vand, hvorpå opløsningerne blandes. Der fås en omtrent
mættet opløsning af sølvacetat. At denne desuden indeholder Na+ og NO3- spiller
ingen rolle. Nogle få ml tages fra for at kontrollere at der ikke sker
udfældning i blandingen.
Resten deles i 2 lige store dele. Den ene fældes med 2,0 g AgNO3, den anden med
1,6 g CH3COONa3H2O **), begge salte opløst i 5 ml vand. Der udfældes CH3COOAg
i begge portioner.

*) eller 0,24 g vandfast salt, **) eller 0,97 g vandfrit salt
