\documentclass[a5paper,20pt]{book}
% Det er her vi sætter margener. Pakken geometry giver også adgang til andet.
\usepackage[a5paper, inner=10mm, outer=10mm, top=10mm, bottom=10mm]{geometry}
% Nødvendigt for at få chemfig til at virke.
\usepackage{tikz}
% Sætter de kemiske formler
\usepackage{chemformula}
% Sikrer at vi kan bruge danske specialtegn
\usepackage[utf8]{inputenc}
% Sætter strukturerne
\usepackage{mol2chemfig}
\usepackage{chemfig}
% Font-encoding, bl.a. det tyske dobbelt-s
\usepackage[T1]{fontenc}


\usepackage{titlesec}

% Hvor vi selv styrer ting
\usepackage{dshk}

\titleformat{\chapter}{\normalfont\huge}{}{20pt}{\huge\bf}
\titlespacing{\chapter}{0pc}{1.5ex plus .1ex minus .2ex}{1pc}
\titlespacing{\chapter}{0pc}{1.5ex plus .1ex minus .2ex}{1pc}


\begin{document}
Skriftstørrelsen skal være 11.
Hvilken font? Vistnok Times Roman-font
Vi skal nok kigge på margener også.

Pakken chemformula giver adgang til:
\ch{3 H2O}

\ch{1/2 H2O}

\ch{AgCl2-}

\ch{H2_{(aq)}}

Se gerne https://tex.stackexchange.com/questions/384610/how-to-write-a-chemical-formula

Vi skal have separat sidenummerering på de første sider med små romertal. Og ingen sidenummerering på den allerførste side.
\pagenumbering{gobble}

\begin{center}
\begin{bf}
\Huge{Kemiske småforsøg}
\end{bf}

\hfill

\hfill

Tilrettelagt og redigeret af

\textit{Asbjørn Petersen \& Christian B. Knudsen}

\vspace*{\fill}
Udgivet af

Dansk Selskab for Historisk Kemi

2019
\end{center}

\pagenumbering{roman}
Kemiske småforsøg
Copyright (c) 2019, Dansk Selskab for Historisk Kemi.
Tryk: Et eller andet sted
Trykt i Danmark 2019
ISBN: tretten cifre

Forside illustration



Dansk Selskab for historisk Kemi
Bestyrelseslisten

\chapter{Forord.}

Bostrups klumme. Interessante forsøg. 100 året for Dansk Kemi. Dansk selskab for historisk kemi.
 Og sådan noget.

\pagenumbering{arabic}
\tableofcontents
Cracking på en anden måde
dansk kemi vol 61 iss 1 p 34

Forfatter:

Cracking af paraffinolie eller andre tunge råoliefractioner er et velkendt
forsøg i skolen. Her skal der beskrives en enklere og morsom udførelse af
forsøget.

I et supremax reagensglas hældes 3 ml paraffinolie, hvorefter der stoppes
rockwool i glasset til al paraffinolie er opsuget. Tørret perlekatalysator
anbringes ovenpå. reagensglasset opspændes vandret og forbindes via et kølet
forlag med en "gasbeholder". perlekatalysatoren ophedes med en fiskehalebrænder,
som efter 1 minut flyttes 1 cm til venstre, så paraffinolien også bliver varmet
op. Den dannede gas samles i gasbeholderen. I forlaget samles en væske, som kan
karakteriseres på sædvanlig måde. Gassen brænder med en lysende sodende flamme
p.g.a. alkenindholdet, men ledes gassen først gennem et glasrør med lidt
platin-asbest, vil forbrændingen ske uden sodning, da alkenerne reagerer med
gassens hydrogen under dannelse af alkaner. Beskyttelsesbriller!

Crackgassens sammensætning

Apparatur og kemikalier: 2 stk. 100 ml glasstempel ()

\emne{Emil Petersen – og et småforsøg}


\danskkemi{Dansk Kemi 2006, 87(5) p 46}



»Kobbersulfat prøves for Jern; 2 Gr. opløses i Vand, der tilsættes lidt Salpetersyre og Opløsningen inddampes. Derefter overmættes den med Ammoniak og fi ltreres; et Indhold af Jern vil
da vise sig som en rødbrun Rest af Jerntveiltehydrat paa Filtret«.
Ovenstående småforsøg er beskrevet af Emil Petersen og offentliggjort i hans Titreranalytiske Methoder, der udkom i 1905.
Emil Petersen blev født den 12. april 1856. Han blev døbt
Christian Emil Ulrich Petersen, men undlod altid mellemnavnene og kaldte sig Emil Petersen – så det vil vi også gøre her.
Emil Petersen blev optaget på Den Polytekniske Læreanstalt
og bestod kandidateksamen i Anvendt Naturvidenskab i 1879.
Han var derefter en kort tid ansat ved Elghammars Järnvärk i
Småland, hvor han udarbejdede en metode til fremstilling af
vanadiumpræparater af slaggen fra malm fra Taberg. Han rejste
tilbage til Danmark og bestod studentereksamen i 1881. I 1888
blev han dr.phil. på afhandlingen Vanadinet og dets nærmeste
Analoger.

I 1870-1872 var han frivillig lærling i marinen, 1872-1874
elev på søofficersskolen. 1881-1882 assistent for S.M. Jørgensen
ved Kemisk Laboratorium på Den Polytekniske Læreanstalt og
1882-1883 i Paris, hvor han studerede naturvidenskabshistorie.

Efter hjemkomsten skrev han en artikelserie i »Tilskueren« om
franske naturvidenskabsmænd. 1885-1893 assistent ved Universitetets kemiske laboratorium, 1895-1901 docent og 1901-1907
professor samme sted.

Hans helbred var ikke godt: Den 2. juli 1907 døde han 51 år
gammel – alt for ung. Af mange historikere omtales han som
ham, der indførte Den Fysiske Kemi i Danmark.

\emne{Småforsøg med protolytiske reaktioner i vandige saltopløsninger}

\danskkemi{87 2 18 }

\deloverskrift{Saltes reaktion med vand.}


De anførte forsøg er en del af en samling, der blev udviklet under ledelse af R.W. Asmussen på Kemisk Laboratorium B på DTU i 1950'erne

Den protolytiske tilstand i opløsninger af salte er bestemt af de tilstedeværende ioners protolytiske karakter.

Ved brug af protolytiske indikatorer kan man danne sig et skøn over vandige saltopløsningers pH-værdi. De protolytiske indikatorer skifter farve i et for hver indikator karakteristisk område.


\deloverskrift{Forberedelse}

Ved forsøgene får man brug for

indikatorer: methylorange $\cdotp$ bromthymolblåt $\cdotp$ phenolphthalein

salte: jernalun $\cdotp$ aluminiumnitrat $\cdotp$ natriumchlorid
       natriumhydrogencarbonat $\cdotp$ natriumcarbonat
                                natriumsulfid



\deloverskrift{Fremgangsmåde}

Følgende opløsninger fremstilles:

1. 4 reagensglas fyldes halvt med vand, og 4-5 dråber methylorange tilsættes. Til hvert enkelt glas sættes hhv. én spatelfuld 1) jernalun, 2) aluminiumnitrat, 3) natriumhydrogencarbonat og 4) natriumchlorid.

2. Til 4 reagensglas med vand og 4-5 dråber bromthymolblåt sættes hhv. én spatelfuld 1) aluminiumnitrat, 2) natriumchlorid, 3) natriumhydrogencarbonat og 4) natriumcarbonat.

3. Til 4 reagensglas med vand og 4-5 dråber phenolphthalein sætets hhv. en spatelfuld 1) natriumchlorid, 2) natriumhydrogencarbonat, 3) natriumcarbonat og 4) natriumsulfid.



Litteratur

R.W. Asmussen m.fl. 1955: Vejledning til øvelser i Kemi for M, B og E. Polyteknisk Forening: 47.

\emne{Bittermandelolie}

\danskkemi{85 nr. 5, s. 46}

Nogle dråber bittermandelolie (bemærk duften af mandler) hældes på en glasplade, der befinder sig på en tændt overheadprojektor. Billedet af dråben fokuseres på et lærred.
Efter få minutters forløb kan man se, at der af olien dannes et fast stof. Efter en time er al olie omdannet til et krystallinsk hvidt stof. Samtidig er lugten af mandler forsvundet [2].

\deloverskrift{Opdagelseshistorie}

I det Kongelige Videnskabernes Selskab holdt professor H.A. Vogel i Berlin 1817 foredrag om en række kemiske forsøg, han havde udført med bitre mandler. Han havde knust dem til et pulver og derefter hældt vand på.
Blandingen havde han underkastet en destillation. I forlaget, der var kølet med sne, fik han opsamlet en blanding af en olie og vand. Da væskerne ikke var blandbare, var det ikkesvært at skille den tungere olie fra det lettere vand. Olien fik naturligt nok navnet bittermandelolie.
Vogel udførte en række forsøg med bittermandelolie. Han var især stolt af at have opdaget, at bittermandelolie ved at henligge i luften udsat for lys blev til et krystallinsk stof [1].

\deloverskrift{Benzoesyre}
Carl Heinrich Stange viste i 1823, at det hvide krystallinske stof, der dannedes af bittermandelolien, var benzoesyre.
Justus von Liebig og Friedrich Wöhler blev interesserede i det åbenbare slægtskab mellem bittermandelolie og benzoesyre. I 1832 begyndte de i Gießen at udføre en række forsøg, der viste, at der eksisterede en hel klasse af stoffer, der tilhørte samme familie.
Ikke blot bittermandelolien men også benzoylchlorid, benzoylbromid, benzamid og benzoylsulfid indeholdt den samme gruppe af atomer, som fik navnet benzoyl [3].

\deloverskrift{Moderne beskrivelse}
Bittermandelolie er benzaldehyd \ch{C6H5CHO}. Det faste stof, der dannes, er benzoesyre \ch{C6H5COOH}. Vogels forsøg, der i forenklet form er beskrevet ovenfor, kan beskrives ved reaktionsskemaet

\ch{2 C6H5CHO + O2 + sollys = 2 C6H5COOH}

\chemfig{
                   O% 8
            =[:270]% 7
                      (
                -[:330]% 9
                      )
            -[:210]% 3
            -[:270]% 2
            -[:210]% 1
                      (
    -[:90,,,,draw=none]\mcfcringle{1.3}% (o)
                      )
            -[:150]% 6
             -[:90]% 5
             -[:30]% 4
                      (
                -[:330]% -> 3
                      )
}

Litteratur
H.A. Vogel, 1817: Versuche über die bittern Mandeln. J.Chemie und Physik 20: 59-74
O.Bostrup, P.Kjeldsen 1993, Organisk kemiske reaktioner (Herning: Systime): 39
W. Strube 1981: Der historische Weg der Chemie 2 (Leipzib: VEB):44


\emne{Triboluminescens}

\danskkemi{Dansk kemi 86, nr. 8 2005, p 37}

Tribologi er læren om gnidning og smøring, teori og teknik. Luminescens er betegnelsen for lysudsendelse, der ikke skyldes glødning og høj temperatur. Triboluminescens er følgelig den del af naturvidenskaben, hvor lysudsendelse ved gnidning behandles.
Der er ikke mange stoffer, der udviser Triboluminescens. I det følgende skal der gives en opskrift på fremstillingen af et af disse sjældne eksempler.

\deloverskrift{Fremstilling af et Triboluminescerende stof.}
Udgangspunktet er 2-aminobenzoesyre, der også kaldes anthranilsyre. Strukturen er vist på figuren. 10 g af dette stof hældes i en 100 mL kolbe, der forsynes med en tilbagesvaler.
Tilsæt 30 mL eddikesyreanydrid og opvarm forsigtigt til kolbens indhold kooger.
Blandingen afkøles i 15 min.
Afkøling
Tilsæt 10 mL vand og opvarm atter til kogning, og afkøl igen kolbe med indhold.
Bundfaldet af 2-ethanoylamninobenzoesyre filtreres fra og vaskes med en lille smule iskold methanol.
Bundfaldet lufttørres.

\deloverskrift{Påvisning af Triboluminescens}

Mellem to urglas anbringes krystaller af det fremstillede stof. I et mørklagt lokale gnides de to glas imod hinanden, og man iagttager Luminescensen.

\chemfig{
            O% 8
     =[:270]% 7
               (
     -[:330,,,1]OH% 9
               )
     -[:210]% 3
    =_[:270]% 2
               (
     -[:330,,,1]NH_2% 10
               )
     -[:210]% 1
    =_[:150]% 6
      -[:90]% 5
     =_[:30]% 4
               (
         -[:330]% -> 3
               )
}

\chemfig{
            O% 8
     =[:270]% 7
               (
     -[:330,,,1]OH% 9
               )
     -[:210]% 3
    =_[:270]% 2
               (
     -[:330,,,1]NH% 10
      -[:270,,1]% 11
                   (
             =[:205]O% 13
                   )
         -[:320]% 12
               )
     -[:210]% 1
    =_[:150]% 6
      -[:90]% 5
     =_[:30]% 4
               (
         -[:330]% -> 3
               )
}

\emne{"Bjerrums forsøg"}
\danskkemi{Dansk Kemi 86, 3, 2005, p. 40}

Af Børge Riis Larsen

Kemikeren Niels Bjerrums navn møder vi i kemilærebøgerne i forbindelse med de
såkaldte Bjerrum-diagrammer. Men han var også ophavsmand til andet. Det var
eksempelvis ham, der foreslog at logaritmere syrestyrkekonstanten $K_{s}$
til $pK_{s}$.
Han viste at aminosyrer eksisterer som amfoioner og dermed har saltkarakter.
Desuden begrundede han, at stærke elektrolytter i fortyndet vandig opløsning
er fuldstændig dissocierede i ioner. Det er beskrevet i et kapitel i den netop
udkomne bog om Niels Bjerrum [1].

Ved præsentationen heraf den 19. november 2004 på H.C. Ørsted Institutet viste
jeg "Bjerrums forsøg", som beskrives i det følgende.

Som nævnt var det Bjerrum, der fremsatte påstanden om elektrolytternes
fuldstændige dissociation. Man har i mange år ment, at det skete ved et
kemikermøde i London i 1909; men det skete faktisk året før ved en af de
forelæsninger, han holdt i forbindelse med konkurrencen om et professorat i
kemi. Her dystede han mod sin gamle skole- og studiekammerat J.N. Brønsted [2].

Påstanden, har Tovborg Jensen skrevet, kan indses at være korrekt blot ved at
betragte tre lige fortyndede opløsninger af kobber(II)chlorid, kobber(II)sulfat
og kobber(II)nitrat og konstatere, at de har helt samme farve. Det må så betyde,
at de indeholder samme mængde af samme kobberforbindelse - den frie hydratiserede
kobber(II)-ion. Bjerrum havde tidligere vist, at to forskellige ioner eller
molekyler aldrig blot tilnærmelsesvis har samme farve [3].


\deloverskrift{Eksperimentelt}


Man afvejer

1. 1,25 g (5,0 mmol) kobber(II)sulfat-vand (1/5), \ch{CuSO4}$\cdotp$\ch{5 H2O}

2. 1,21 g (5,0 mmol) kobber(II)nitrat-vand (1/3), \ch{Cu(NO3)2}$\cdotp$\ch{3 H2O}

3. 0.85 g (5,0 mmol) kobber(II)chlorid-vand (1/2), \ch{CuCl2}$\cdotp$\ch{2 H2O}

Man konstaterer, at de tre forbindelser har forskellige farver.

Dernæst overføres de tre salte til tre 50 mL målekolber, som fyldes til stregen
med demineraliseret vand. Man vil nu - efter at det hele er opløst - iagttage,
at de tre opløsninger har præcis samme farve.

Referencer
1. B. Riis Larsen: Bjerrum og skolen. I A. Kildebæk Nielsen (red.): Niels Bjerrum
(1879-1958). Liv og værk. Udgivet som Historisk-kemiske skrifter nr. 15 af Dansk
Selskab for Historisk Kemi (2004).
2. Om deres skoletid i Metropolitanskolen i 1890'erne henviser jeg til mit
kapitel Brønsted, Bjerrum og Metropolitanskolen p. 30-40 i: B. Riis Larsen (red.):
J.N. Brønsted - en dansk kemiker udgivet som Historisk-kemiske skrifter nr. 8
af Dansk Selskab for Historisk Kemi (1997).
3. A. Tovborg Jensen: C.T. Barfoed, Odin T. Christensen og Niels Bjerrum. Tre
kemikere ved den Kongelige Veterinær- og Landbohøjskole i København 1858-1949. I:
B. Jerslev (red.): Kemien i Danmark III. Danske Kemikere. (Kbh. 1968 - Nyt Nordisk
Forlag. Arnold Busck).

\end{document}
