\documentclass[a5paper,20pt]{book}
% Det er her vi sætter margener. Pakken geometry giver også adgang til andet.
\usepackage[a5paper, inner=10mm, outer=10mm, top=10mm, bottom=10mm]{geometry}
% Nødvendigt for at få chemfig til at virke.
\usepackage{tikz}
% Sætter de kemiske formler
\usepackage{chemformula}
% Sikrer at vi kan bruge danske specialtegn
\usepackage[utf8]{inputenc}
% Sætter strukturerne
\usepackage{mol2chemfig}
\usepackage{chemfig}
% Font-encoding, bl.a. det tyske dobbelt-s
\usepackage[T1]{fontenc}
% for at kunne inkludere pdf'ere
\usepackage{pdfpages}
% For at kunne styre sidelayhout og den slags.
\usepackage{titlesec}

\usepackage{longtable}

% Hvor vi selv styrer ting
\usepackage{dshk}


\begin{document}

\pagenumbering{gobble}

\begin{center}
\begin{bf}
\Huge{Kemiske småforsøg}
\end{bf}

\hfill

\hfill

Tilrettelagt og redigeret af

\textit{Asbjørn Petersen \& Christian B. Knudsen}

\vspace*{\fill}
Udgivet af

Dansk Selskab for Historisk Kemi

2019
\end{center}

\pagenumbering{roman}
Kemiske småforsøg
Copyright (c) 2019, Dansk Selskab for Historisk Kemi.
Tryk: Et eller andet sted
Trykt i Danmark 2019
ISBN: tretten cifre

Forside illustration



Dansk Selskab for historisk Kemi
Bestyrelseslisten

\chapter{Forord.}

Bostrups klumme. Interessante forsøg. 100 året for Dansk Kemi. Dansk selskab for historisk kemi.
 Og sådan noget.

\pagenumbering{arabic}
\tableofcontents

\includepdf[pages=-]{pdfs/1980-61-1-34.pdf}
\emne{Cracking på en anden måde}
\danskkemi{ dansk kemi vol 61 iss 1 p 34}
\forfatter{Peter Norrild}



Cracking af paraffinolie eller andre tunge råoliefractioner er et velkendt
forsøg i skolen. Her skal der beskrives en enklere og morsom udførelse af
forsøget.

I et supremax reagensglas hældes 3 ml paraffinolie, hvorefter der stoppes
rockwool i glasset til al paraffinolie er opsuget. Tørret perlekatalysator
anbringes ovenpå. reagensglasset opspændes vandret og forbindes via et kølet
forlag med en "gasbeholder". perlekatalysatoren ophedes med en fiskehalebrænder,
som efter 1 minut flyttes 1 cm til venstre, så paraffinolien også bliver varmet
op. Den dannede gas samles i gasbeholderen. I forlaget samles en væske, som kan
karakteriseres på sædvanlig måde. Gassen brænder med en lysende sodende flamme
p.g.a. alkenindholdet, men ledes gassen først gennem et glasrør med lidt
platin-asbest, vil forbrændingen ske uden sodning, da alkenerne reagerer med
gassens hydrogen under dannelse af alkaner. Beskyttelsesbriller!

\deloverskrift{Crackgassens sammensætning}

Apparatur og kemikalier: 2 stk. 100 ml glasstempel (Exelo), glasrør,
gummislange, tregangshane, hydrogen, 1-buten (Fluka), platin-asbest
eller platin-aktivkul.
I apparaturet, der er vist på tegningen, udføres følgende bestemmelser.
A: Apparaturet gennemskylles med crackgas. Herefter fyldes der nøjagtigt
100 ml crackgas i stempel 1. Gassen ledes over katalysatoren til konstant
volumen opnås. Volumen formindskelsen -Delta Va noteres.

R-CH=CH2 + H2 -> R-CH2-CH3
1 vol 1 vol 1 vol

Volumen formindskelsen er et udtryk for indholdet af den komponent, der
er i underskud - altså enten alekn eller hydrogen. Der udføres derfor to bestemmelser mere.
B: Til stempel 2 sættes 50 ml H2 og til stempel 1 100 ml crackgas.
Gassernes blandes og ledes over katalysatoren. Volumen formindskelsen
-Delta Vb noteres.
C: Til stempel 2 sættes 50 ml buten og til stempel 1 100 ml
crackgas. Gasserne blandes og ledes over katalysatoren. - Delta Vc
noteres.
Måleeksempel:
- Delta Va = 32 ml Hydrogen
-Delta Vb = 44 ml Alken
-Delta Vc = 33 ml Hydrogen

I dette eksempel er vol.forholdet H2/alken/alkan = 32/44/24.
Forsøg C har ikke været nødvendigt i dette tilfælde, men gassens
sammensætning varierer en del fra forsøg til forsøg.

noter til illustration 1:
A: Paraffinolie i rockwool, B: Tørret perlekatalysator D: Kølet forlag,
D: Gasbeholder (plastflaske), E: Klemme F: Spidset glasrør.

Noter til illustration 2:
A: Glasrør med Pt-katalysator,
B: tregangshane

\include{dele/62-3-a}
\includepdf[pages=-]{pdfs/2004-85-5-46.pdf}
\emne{Bittermandelolie}


\danskkemi{85 nr. 5, s. 46}

Nogle dråber bittermandelolie (bemærk duften af mandler) hældes på en glasplade, der befinder sig på en tændt overheadprojektor. Billedet af dråben fokuseres på et lærred.
Efter få minutters forløb kan man se, at der af olien dannes et fast stof. Efter en time er al olie omdannet til et krystallinsk hvidt stof. Samtidig er lugten af mandler forsvundet [2].

\deloverskrift{Opdagelseshistorie}

I det Kongelige Videnskabernes Selskab holdt professor H.A. Vogel i Berlin 1817 foredrag om en række kemiske forsøg, han havde udført med bitre mandler. Han havde knust dem til et pulver og derefter hældt vand på.
Blandingen havde han underkastet en destillation. I forlaget, der var kølet med sne, fik han opsamlet en blanding af en olie og vand. Da væskerne ikke var blandbare, var det ikkesvært at skille den tungere olie fra det lettere vand. Olien fik naturligt nok navnet bittermandelolie.
Vogel udførte en række forsøg med bittermandelolie. Han var især stolt af at have opdaget, at bittermandelolie ved at henligge i luften udsat for lys blev til et krystallinsk stof [1].

\deloverskrift{Benzoesyre}
Carl Heinrich Stange viste i 1823, at det hvide krystallinske stof, der dannedes af bittermandelolien, var benzoesyre.
Justus von Liebig og Friedrich Wöhler blev interesserede i det åbenbare slægtskab mellem bittermandelolie og benzoesyre. I 1832 begyndte de i Gießen at udføre en række forsøg, der viste, at der eksisterede en hel klasse af stoffer, der tilhørte samme familie.
Ikke blot bittermandelolien men også benzoylchlorid, benzoylbromid, benzamid og benzoylsulfid indeholdt den samme gruppe af atomer, som fik navnet benzoyl [3].

\deloverskrift{Moderne beskrivelse}
Bittermandelolie er benzaldehyd \ch{C6H5CHO}. Det faste stof, der dannes, er benzoesyre \ch{C6H5COOH}. Vogels forsøg, der i forenklet form er beskrevet ovenfor, kan beskrives ved reaktionsskemaet

\ch{2 C6H5CHO + O2 + sollys = 2 C6H5COOH}

\chemfig{
                   O% 8
            =[:270]% 7
                      (
                -[:330]% 9
                      )
            -[:210]% 3
            -[:270]% 2
            -[:210]% 1
                      (
    -[:90,,,,draw=none]\mcfcringle{1.3}% (o)
                      )
            -[:150]% 6
             -[:90]% 5
             -[:30]% 4
                      (
                -[:330]% -> 3
                      )
}

Litteratur
H.A. Vogel, 1817: Versuche über die bittern Mandeln. J.Chemie und Physik 20: 59-74
O.Bostrup, P.Kjeldsen 1993, Organisk kemiske reaktioner (Herning: Systime): 39
W. Strube 1981: Der historische Weg der Chemie 2 (Leipzib: VEB):44

\include{dele/86-8}
%\includepdf[pages=-]{pdfs/.pdf}
\emne{Småforsøg med protolytiske reaktioner i vandige saltopløsninger}

\danskkemi{87 2 18 }

\deloverskrift{Saltes reaktion med vand.}


De anførte forsøg er en del af en samling, der blev udviklet under ledelse af R.W. Asmussen på Kemisk Laboratorium B på DTU i 1950'erne

Den protolytiske tilstand i opløsninger af salte er bestemt af de tilstedeværende ioners protolytiske karakter.

Ved brug af protolytiske indikatorer kan man danne sig et skøn over vandige saltopløsningers pH-værdi. De protolytiske indikatorer skifter farve i et for hver indikator karakteristisk område.


\deloverskrift{Forberedelse}

Ved forsøgene får man brug for

indikatorer: methylorange $\cdotp$ bromthymolblåt $\cdotp$ phenolphthalein

salte: jernalun $\cdotp$ aluminiumnitrat $\cdotp$ natriumchlorid
       natriumhydrogencarbonat $\cdotp$ natriumcarbonat
                                natriumsulfid



\deloverskrift{Fremgangsmåde}

Følgende opløsninger fremstilles:

1. 4 reagensglas fyldes halvt med vand, og 4-5 dråber methylorange tilsættes. Til hvert enkelt glas sættes hhv. én spatelfuld 1) jernalun, 2) aluminiumnitrat, 3) natriumhydrogencarbonat og 4) natriumchlorid.

2. Til 4 reagensglas med vand og 4-5 dråber bromthymolblåt sættes hhv. én spatelfuld 1) aluminiumnitrat, 2) natriumchlorid, 3) natriumhydrogencarbonat og 4) natriumcarbonat.

3. Til 4 reagensglas med vand og 4-5 dråber phenolphthalein sætets hhv. en spatelfuld 1) natriumchlorid, 2) natriumhydrogencarbonat, 3) natriumcarbonat og 4) natriumsulfid.



Litteratur

R.W. Asmussen m.fl. 1955: Vejledning til øvelser i Kemi for M, B og E. Polyteknisk Forening: 47.

%\includepdf[pages=-]{pdfs/.pdf}
\emne{Emil Petersen – og et småforsøg}


\danskkemi{Dansk Kemi 2006, 87(5) p 46}



»Kobbersulfat prøves for Jern; 2 Gr. opløses i Vand, der tilsættes lidt Salpetersyre og Opløsningen inddampes. Derefter overmættes den med Ammoniak og fi ltreres; et Indhold af Jern vil
da vise sig som en rødbrun Rest af Jerntveiltehydrat paa Filtret«.
Ovenstående småforsøg er beskrevet af Emil Petersen og offentliggjort i hans Titreranalytiske Methoder, der udkom i 1905.
Emil Petersen blev født den 12. april 1856. Han blev døbt
Christian Emil Ulrich Petersen, men undlod altid mellemnavnene og kaldte sig Emil Petersen – så det vil vi også gøre her.
Emil Petersen blev optaget på Den Polytekniske Læreanstalt
og bestod kandidateksamen i Anvendt Naturvidenskab i 1879.
Han var derefter en kort tid ansat ved Elghammars Järnvärk i
Småland, hvor han udarbejdede en metode til fremstilling af
vanadiumpræparater af slaggen fra malm fra Taberg. Han rejste
tilbage til Danmark og bestod studentereksamen i 1881. I 1888
blev han dr.phil. på afhandlingen Vanadinet og dets nærmeste
Analoger.

I 1870-1872 var han frivillig lærling i marinen, 1872-1874
elev på søofficersskolen. 1881-1882 assistent for S.M. Jørgensen
ved Kemisk Laboratorium på Den Polytekniske Læreanstalt og
1882-1883 i Paris, hvor han studerede naturvidenskabshistorie.

Efter hjemkomsten skrev han en artikelserie i »Tilskueren« om
franske naturvidenskabsmænd. 1885-1893 assistent ved Universitetets kemiske laboratorium, 1895-1901 docent og 1901-1907
professor samme sted.

Hans helbred var ikke godt: Den 2. juli 1907 døde han 51 år
gammel – alt for ung. Af mange historikere omtales han som
ham, der indførte Den Fysiske Kemi i Danmark.


\chapter{Diverse administrativt}
Skriftstørrelsen skal være 11.
Hvilken font? Vistnok Times Roman-font
Vi skal nok kigge på margener også.

Pakken chemformula giver adgang til:
\ch{3 H2O}

\ch{1/2 H2O}

\ch{AgCl2-}

\ch{H2_{(aq)}}

Se gerne https://tex.stackexchange.com/questions/384610/how-to-write-a-chemical-formula

Vi skal have separat sidenummerering på de første sider med små romertal. Og ingen sidenummerering på den allerførste side.

\begin{longtable}{ |l|l|l|l|l|l|l|l| }
Index & År & Årgang & Hæfte & Side & PDF/Link & tex & Note \\
\hline
\endhead % all the lines above this will be repeated on every page
  1 & 2018 & 99 & 8     &      NA &  &  & \\
  2 & 2018 & 99 & 7     &      NA &  &  & \\
  3 & 2018 & 99 & 6     &      NA &  &  & \\
  4 & 2018 & 99 & 5     &      NA &  &  & \\
  5 & 2018 & 99 & 4     &      NA &  &  & \\
  6 & 2018 & 99 & 3     &      NA &  &  & \\
  7 & 2018 & 99 & 2     &      NA &  &  & \\
  8 & 2018 & 99 & 1     &      NA &  &  & \\
  9 & 2017 & 98 & 11/12 &      NA &  &  & \\
 10 & 2017 & 98 & 10    &      NA &  &  & \\
 11 & 2017 & 98 & 9     &      NA &  &  & \\
 12 & 2017 & 98 & 8     &      NA &  &  & \\
 13 & 2017 & 98 & 6/7   &      NA &  &  & \\
 14 & 2017 & 98 & 5     &      NA &  &  & \\
 15 & 2017 & 98 & 4     &      NA &  &  & \\
 16 & 2017 & 98 & 3     &      NA &  &  & \\
 17 & 2017 & 98 & 1/2   &      NA &  &  & \\
 18 & 2016 & 97 & 12    &      NA &  &  & \\
 19 & 2016 & 97 &    11 &      NA &  &  & \\
 20 & 2016 & 97 &    10 &      NA &  &  & \\
 21 & 2016 & 97 &     9 &      NA &  &  & \\
 22 & 2016 & 97 &     8 &      NA &  &  & \\
 23 & 2016 & 97 &   6/7 &      NA &  &  & \\
 24 & 2016 & 97 &     5 &      NA &  &  & \\
 25 & 2016 & 97 &     4 &      NA &  &  & \\
 26 & 2016 & 97 &     3 &      NA &  &  & \\
 27 & 2016 & 97 &   1/2 &      NA &  &  & \\
 28 & 2015 & 96 &    12 &      NA &  &  & \\
 29 & 2015 & 96 &    11 &      NA &  &  & \\
 30 & 2015 & 96 &    10 &      NA &  &  & \\
 31 & 2015 & 96 &     9 &      NA &  &  & \\
 32 & 2015 & 96 &     8 &      NA &  &  & \\
 33 & 2015 & 96 &   6/7 &      NA  &  &  & \\
 34 & 2015 & 96 &     5 &      NA &  &  & \\
 35 & 2015 & 96 &     4 &      NA &  &  & \\
 36 & 2015 & 96 &     3 &      NA &  &  & \\
 37 & 2015 & 96 &   1/2 &      NA &  &  & \\
 38 & 2014 & 95 &    12 &      NA &  &  & \\
 39 & 2014 & 95 &    11 &      NA &  &  & \\
 40 & 2014 & 95 &    10 &      NA &  &  & \\
 41 & 2014 & 95 &     9 &      NA &  &  & \\
 42 & 2014 & 95 &     8 &         &  &  & \\
 43 & 2014 & 95 &   6/7 &         &  &  & \\
 44 & 2014 & 95 &     5 &         &  &  & \\
 45 & 2014 & 95 &     4 &         &  &  & \\
 46 & 2014 & 95 &     3 &         &  &  & \\
 47 & 2014 & 95 &   1/2 &         &  &  & \\
 48 & 2013 & 94 &    12 &         &  &   & \\
 49 & 2013 & 94 &    11 &         &  &  & \\
 50 & 2013 & 94 &    10 &         &  &  & \\
 51 & 2013 & 94 &     9 &         &  &  & \\
 52 & 2013 & 94 &     8 &         &  &  & \\
 53 & 2013 & 94 &   6/7 &         &  &  & \\
 54 & 2013 & 94 &     5 &         &  &  & \\
 55 & 2013 & 94 &     4 &         &  &  & \\
 56 & 2013 & 94 & 3 sup &         &  &  & \\
 57 & 2013 & 94 &     3 &         &  &  & \\
 58 & 2013 & 94 &   1/2 &         &  &  & \\
 59 & 2012 & 93 &    12 &         &  &  & \\
 60 & 2012 & 93 & 11 tl &         &  &  & \\
 61 & 2012 & 93 & 11    &         &  &  & \\
 62 & 2012 & 93 & 10    &         &  &   & \\
 63 & 2012 & 93 & 8     &         &  &  & \\
 64 & 2012 & 93 & 9     &         &  &  & \\
 65 & 2012 & 93 &   6/7 &         &  &  & \\
 66 & 2012 & 93 &     5 &         &  &  & \\
 67 & 2012 & 93 &     4 &         &  &  & \\
 68 & 2012 & 93 &     3 &         &  &  & \\
 69 & 2012 & 93 & 1/2tll&         &  &  & \\
 70 & 2012 & 93 &   1/2 &         &  &  & \\
 71 & 2011 & 92 &    12 &         &  &  & \\
 72 & 2011 & 92 &    11 &         &  &  & \\
 73 & 2011 & 92 &    10 &         &  &  & \\
 74 & 2011 & 92 &     9 &         &  &  & \\
 75 & 2011 & 92 &     8 &         &  &  & \\
 76 & 2011 & 92 &   6/7 &         &  &  & \\
 77 & 2011 & 92 &     5 &         &  &  & \\
 78 & 2011 & 92 &     4 &         &  &  & \\
 79 & 2011 & 92 &     3 &         &  &  & \\
 80 & 2011 & 92 &   1/2 &         &  &  & \\
 81 & 2010 & 91 &   12  &      NA &  &  & \\
 82 & 2010 & 91 &    11 &         &  &  & \\
 83 & 2010 & 91 &    10 &         &  &  & \\
 84 & 2010 & 91 &     9 &         &  &  & \\
 85 & 2010 & 91 &     8 &         &  &  & \\
 86 & 2010 & 91 &   6/7 &         &  &  & \\
 87 & 2010 & 91 &     5 &         &  &  & \\
 88 & 2010 & 91 &     4 &         &  &  & \\
 89 & 2010 & 91 &     3 &         &  &  & \\
 90 & 2010 & 91 &   1/2 &      NA &  &  & \\
 91 & 2009 & 90 &    12 &      NA &  &  & \\
 92 & 2009 & 90 &    11 &         &  &  & \\
 93 & 2009 & 90 &    10 &         &  &  & \\
 94 & 2009 & 90 &     9 &         &  &  & \\
 95 & 2009 & 90 &     8 &         &  &  & \\
 96 & 2009 & 90 &   6/7 &         &  &  & \\
 97 & 2009 & 90 &     5 &         &  &  & \\
 98 & 2009 & 90 &     4 &      NA &  &  & \\
 99 & 2009 & 90 &     3 &      NA &  &  & \\
100 & 2009 & 90 &     2 &         &  &  & \\
101 & 2009 & 90 &     1 &         &  &  & \\
102 & 2008 & 89 &    12 &      NA &  &  & \\
103 & 2008 & 89 &    11 &         &  &  & \\
104 & 2008 & 89 &    10 &      NA &  &  & \\
105 & 2008 & 89 &     9 &      NA &  &  & \\
106 & 2008 & 89 &     8 &      NA &  &  & \\
107 & 2008 & 89 &   6/7 &      NA &  &  & \\
108 & 2008 & 89 &     5 &      NA &  &  & \\
109 & 2008 & 89 &     4 &      NA &  &  & \\
110 & 2008 & 89 &     3 &         &  & Ej online? eksisterer det? & \\
111 & 2008 & 89 &     2 &      NA &  &  & \\
112 & 2008 & 89 &     1 &      NA &  &  & \\
113 & 2007 & 88 &    12 &      NA &  &  & \\
114 & 2007 & 88 &    11 &      NA &  &  & \\
115 & 2007 & 88 &    10 &      NA &  &  & \\
116 & 2007 & 88 &     9 &      NA &  &  & \\
117 & 2007 & 88 &     8 &      NA &  &  & \\
118 & 2007 & 88 &   6/7 &      NA &  &  & \\
119 & 2007 & 88 &     5 &      NA &  &  & \\
120 & 2007 & 88 &     4 &      NA &  &  & \\
121 & 2007 & 88 &     3 &      NA &  &  & \\
122 & 2007 & 88 &     2 &      NA &  &  & \\
123 & 2007 & 88 &     1 &      NA &  &  & \\
124 & 2006 & 87 &    12 &      NA &  &  & \\
125 & 2006 & 87 &    11 &      NA &  &  & \\
126 & 2006 & 87 &    10 &      NA &  &  & \\
127 & 2006 & 87 &     9 &      65 & https://ipaper.ipapercms.dk/TechMedia/DanskKemi/2006/9/ &  & \\
128 & 2006 & 87 &     8 &      31 & https://ipaper.ipapercms.dk/TechMedia/DanskKemi/2006/8/ &  & \\
129 & 2006 & 87 &   6/7 &      NA &  &  & \\
130 & 2006 & 87 &     5 &      46 & https://ipaper.ipapercms.dk/TechMedia/DanskKemi/2006/5/ &  & \\
131 & 2006 & 87 &     4 &      NA &  &  & \\
132 & 2006 & 87 &     3 &      38 & 2006-87-3-38 &  & \\
133 & 2006 & 87 &     2 &      18 & https://ipaper.ipapercms.dk/TechMedia/DanskKemi/2006/2/ & 87-2  & \\
134 & 2006 & 87 &     1 &      38 & 2006-87-1-38 &  & \\
135 & 2005 & 86 &    12 &      32 & 2005-86-12-32 &  & \\
136 & 2005 & 86 &    11 &      NA &  &  & \\
137 & 2005 & 86 &    10 &      NA &  &  & \\
138 & 2005 & 86 &     9 &         &  &  & \\
139 & 2005 & 86 &     8 &      37 & 2005-86-8-37 & 86-8  & \\
140 & 2005 & 86 &   6/7 &      39 &  &  & \\
141 & 2005 & 86 &     5 &         &  &  & \\
142 & 2005 & 86 &     4 &         & &  & \\
143 & 2005 & 86 &     3 &      40 &  2005-86-3-40 & 86-3 & \\
144 & 2005 & 86 &     2 &         &  &  & \\
145 & 2005 & 86 &     1 &         &  &  & \\
146 & 2004 & 85 &    12 &         &  &  & \\
147 & 2004 & 85 &    11 &         &  &  & \\
148 & 2004 & 85 &    10 &      NA &  &  & \\
149 & 2004 & 85 &     9 &      54 &  &  & \\
150 & 2004 & 85 &     8 &      39 &  &  & \\
151 & 2004 & 85 &   6/7 &         &  &  & \\
152 & 2004 & 85 &     5 &      46 & 2005-85-5-46 & 85-5 & \\
153 & 2004 & 85 &     4 &         &  &  & \\
154 & 2004 & 85 &     3 &      40 &  &  & \\
155 & 2004 & 85 &     2 &      27 &  &  & \\
156 & 2004 & 85 &     1 &         &  &  & \\
157 & 2003 & 84 &    12 &         &  &  & \\
158 & 2003 & 84 &    11 &         &  &  & \\
159 & 2003 & 84 &    10 &         &  &  & \\
160 & 2003 & 84 &     9 &         &  &  & \\
161 & 2003 & 84 &     8 &         &  &  & \\
162 & 2003 & 84 &   6/7 &         &  &  & \\
163 & 2003 & 84 &     5 &         &  &  & \\
164 & 2003 & 84 &     4 &         &  &  & \\
165 & 2003 & 84 &     3 &         &  &  & \\
166 & 2003 & 84 &     2 &         &  &  & \\
167 & 2003 & 84 &     1 &         &  &  & \\
168 & 2002 & 83 &    12 &         &  &  & \\
169 & 2002 & 83 &    11 &         &  &  & \\
170 & 2002 & 83 &    10 &         &  &  & \\
171 & 2002 & 83 &     9 &         &  &  & \\
172 & 2002 & 83 &     8 &         &  &  & \\
173 & 2002 & 83 &   6/7 &         &  &  & \\
174 & 2002 & 83 &    5s &         &  &  & \\
175 & 2002 & 83 &     5 &         &  &  & \\
176 & 2002 & 83 &     4 &         &  &  & \\
177 & 2002 & 83 &     3 &         &  &  & \\
178 & 2002 & 83 &     2 &         &  &  & \\
179 & 2002 & 83 &     1 &         &  &  & \\
180 & 2001 & 82 &    12 &         &  &  & \\
181 & 2001 & 82 &    11 &         &  &  & \\
182 & 2001 & 82 &    10 &         &  &  & \\
183 & 2001 & 82 &    9t &         &  &  & \\
184 & 2001 & 82 &     9 &         &  &  & \\
185 & 2001 & 82 &     8 &         &  &  & \\
186 & 2001 & 82 &   6/7 &         &  &  & \\
187 & 2001 & 82 &    5t &         &  &  & \\
188 & 2001 & 82 &     5 &         &  &  & \\
189 & 2001 & 82 &     4 &         &  &  & \\
190 & 2001 & 82 &    3t &         &  &  & \\
191 & 2001 & 82 &     3 &         &  &  & \\
192 & 2001 & 82 &     2 &         &  &  & \\
193 & 2001 & 82 &     1 &         &  &  & \\
194 & 2000 & 81 &    12 &         &  &  & \\
195 & 2000 & 81 &    11 &         &  &  & \\
196 & 2000 & 81 &    10 &         &  &  & \\
197 & 2000 & 81 &    9t &         &  &  & \\
198 & 2000 & 81 &     9 &         &  &  & \\
199 & 2000 & 81 &     8 &         &  &  & \\
200 & 2000 & 81 &   6/7 &         &  &  & \\
201 & 2000 & 81 &     5 &         &  &  & \\
202 & 2000 & 81 &    4t &         &  &  & \\
203 & 2000 & 81 &     4 &         &  &  & \\
204 & 2000 & 81 &    3t &         &  &  & \\
205 & 2000 & 81 &     3 &         &  &  & \\
206 & 2000 & 81 &     2 &         &    &  & \\
207 & 2000 & 81 &     1 &         &  &  & \\
208 & 1999 & 80 &    12 &         &  &  & \\
209 & 1999 & 80 &    11 &         &  &  & \\
210 & 1999 & 80 &    10 &         &  &  & \\
211 & 1999 & 80 &     9 &         &  &  & \\
212 & 1999 & 80 &     8 &         &  &  & \\
213 & 1999 & 80 &   6/7 &         &  &  & \\
214 & 1999 & 80 &     5 &         &  &  & \\
215 & 1999 & 80 &     4 &         &  &  & \\
216 & 1999 & 80 &     3 &         &  &  & \\
217 & 1999 & 80 &     2 &         &  &  & \\
218 & 1999 & 80 &     1 &         &  &  & \\
219 & 1998 & 79 &    12 &         &  &  & \\
220 & 1998 & 79 &    11 &         &  &  & \\
221 & 1998 & 79 &    10 &         &  &  & \\
222 & 1998 & 79 &    9t &         &  &  & \\
223 & 1998 & 79 &     9 &         &  &  & \\
224 & 1998 & 79 &     8 &         &  &  & \\
225 & 1998 & 79 &   6/7 &         &  &  & \\
226 & 1998 & 79 &     5 &         &  &  & \\
227 & 1998 & 79 &     4 &         &  &  & \\
228 & 1998 & 79 &     3 &         &  &  & \\
229 & 1998 & 79 &     2 &         &  &  & \\
230 & 1998 & 79 &     1 &         &  &  & \\
231 & 1997 & 78 &    12 &         &  &  & \\
232 & 1997 & 78 &    11 &         &  &  & \\
233 & 1997 & 78 &    10 &         &  &  & \\
234 & 1997 & 78 &     9 &         &  &  & \\
235 & 1997 & 78 &     8 &         &  &  & \\
236 & 1997 & 78 &   6/7 &         &  &  & \\
237 & 1997 & 78 &     5 &         &  &  & \\
238 & 1997 & 78 &     4 &         &  &  & \\
239 & 1997 & 78 &     3 &         &  &  & \\
240 & 1997 & 78 &     2 &         &  &  & \\
241 & 1997 & 78 &     1 &         &                &  & \\
242 & 1996 & 77 &    12 &         &                &  & \\
243 & 1996 & 77 &    11 &         &                &  & \\
244 & 1996 & 77 &    10 &         &                &  & \\
245 & 1996 & 77 &     9 &         &                &  & \\
246 & 1996 & 77 &     8 &         &                &  & \\
247 & 1996 & 77 &   6/7 &         &                &  & \\
248 & 1996 & 77 &     5 &         &                &  & \\
249 & 1996 & 77 &     4 &         &                &  & \\
250 & 1996 & 77 &     3 &         &                &  & \\
251 & 1996 & 77 &     2 &         &                &  & \\
252 & 1996 & 77 &     1 &         &                &  & \\
253 & 1995 & 76 &    12 &         &                &  & \\
254 & 1995 & 76 &    11 &         &                &  & \\
255 & 1995 & 76 &    10 &         &                &  & \\
256 & 1995 & 76 &     9 &         &                &  & \\
257 & 1995 & 76 &     8 &         &                &  & \\
258 & 1995 & 76 &   6/7 &         &                &  & \\
259 & 1995 & 76 &     5 &         &                &  & \\
260 & 1995 & 76 &     4 &         &                &  & \\
261 & 1995 & 76 &     3 &         &                &  & \\
262 & 1995 & 76 &     2 &         &                &  & \\
263 & 1995 & 76 &     1 &         &                &  & \\
264 & 1994 & 75 &    12 &         &                &  & \\
265 & 1994 & 75 &    11 &         &                &  & \\
266 & 1994 & 75 &    10 &         &                &  & \\
267 & 1994 & 75 &     9 &         &                &  & \\
268 & 1994 & 75 &     8 &         &                &  & \\
269 & 1994 & 75 &   6/7 &         &                &  & \\
270 & 1994 & 75 &     5 &         &                &   & \\
271 & 1994 & 75 &     4 &         &                &  & \\
272 & 1994 & 75 &     3 &         &                &  & \\
273 & 1994 & 75 &     2 &         &                &  & \\
274 & 1994 & 75 &     1 &         &                &  & \\
275 & 1993 & 74 &    12 &         &                &  & \\
276 & 1993 & 74 &    11 &         &                &  & \\
277 & 1993 & 74 &    10 &         &                &  & \\
278 & 1993 & 74 &     9 &         &                &  & \\
279 & 1993 & 74 &     8 &         &                &  & \\
280 & 1993 & 74 &   6/7 &         &                &  & \\
281 & 1993 & 74 &     5 &         &                &  & \\
282 & 1993 & 74 &     4 &         &                &  & \\
283 & 1993 & 74 &     3 &         &                &  & \\
284 & 1993 & 74 &     2 &         &                &  & \\
285 & 1993 & 74 &     1 &         &                &  & \\
286 & 1993 & 74 &   reg &         &                &  & \\
287 & 1992 & 73 &    12 &         &                &  & \\
288 & 1992 & 73 &    11 &         &                &  & \\
289 & 1992 & 73 &    10 &         &                &  & \\
290 & 1992 & 73 &     9 &         &                &  & \\
291 & 1992 & 73 &     8 &         &                &  & \\
292 & 1992 & 73 &   6/7 &         &                &  & \\
293 & 1992 & 73 &     5 &         &                &  & \\
294 & 1992 & 73 &     4 &         &                &  & \\
295 & 1992 & 73 &     3 &         &                &  & \\
296 & 1992 & 73 &     2 &         &                &  & \\
297 & 1992 & 73 &     1 &         &                &  & \\
298 & 1991 & 72 &    12 &         &                &  & \\
299 & 1991 & 72 &    11 &         &                &  & \\
300 & 1991 & 72 &    10 &         &                &  & \\
301 & 1991 & 72 &     9 &         &                &  & \\
302 & 1991 & 72 &     8 &         &                &  & \\
303 & 1991 & 72 &   6/7 &         &                &  & \\
304 & 1991 & 72 &     5 &         &                &  & \\
305 & 1991 & 72 &     4 &         &                &  & \\
306 & 1991 & 72 &     3 &         &                &  & \\
307 & 1991 & 72 &     2 &         &                &  & \\
308 & 1991 & 72 &     1 &         &                &  & \\
309 & 1990 & 71 &    12 &         &                &  & \\
310 & 1990 & 71 &    11 &         &                &  & \\
311 & 1990 & 71 &    10 &         &                &  & \\
312 & 1990 & 71 &     9 &         &                &  & \\
313 & 1990 & 71 &     8 &         &                &  & \\
314 & 1990 & 71 &   6/7 &         &                &  & \\
315 & 1990 & 71 &     5 &         &                &   & \\
316 & 1990 & 71 &     4 &         &                &  & \\
317 & 1990 & ?  &     3 &         &                &  & \\
318 & 1990 & ?  &     2 &         &                &  & \\
319 & 1990 & ?  &     1 &         &                &  & \\
320 & 1989 & 70 &    12 &         &                &   & \\
321 & 1989 & 70 &    11 &         &                &  & \\
322 & 1989 & 70 &    10 &         &                &  & \\
323 & 1989 & 70 &     9 &         &                &  & \\
324 & 1989 & 70 &     8 &         &                &  & \\
325 & 1989 & 70 &   6/7 &         &                &  & \\
326 & 1989 & 70 &     5 &         &                &  & \\
327 & 1989 & 70 &     4 &         &                &  & \\
328 & 1989 & 70 &     3 &         &                &  & \\
329 & 1989 & 70 &     2 &         &                &  & \\
330 & 1989 & 70 &     1 &         &                &  & \\
331 & 1988 & 69 &    12 &         &                &  & \\
332 & 1988 & 69 &    11 &         &                &  & \\
333 & 1988 & 69 &    10 &         &                &  & \\
334 & 1988 & 69 &     9 &         &                &  & \\
335 & 1988 & 69 &     8 &         &                &  & \\
336 & 1988 & 69 &   6/7 &         &                &  & \\
337 & 1988 & 69 &     5 &         &                &  & \\
338 & 1988 & 69 &     4 &         &                &  & \\
339 & 1988 & 69 &     3 &         &                &  & \\
340 & 1988 & 69 &     2 &         &                &  & \\
341 & 1988 & 69 &     1 &         &                &  & \\
342 & 1987 & 68 &    12 &         &                &  & \\
343 & 1987 & 68 &    11 &         &                &  & \\
344 & 1987 & 68 &    10 &         &                &  & \\
345 & 1987 & 68 &     9 &         &                &  & \\
346 & 1987 & 68 &     8 &         &                &  & \\
347 & 1987 & 68 &   6/7 &         &                &  & \\
348 & 1987 & 68 &     5 &         &                &  & \\
349 & 1987 & 68 &     4 &         &                &  & \\
350 & 1987 & 68 &     3 &         &                &  & \\
351 & 1987 & 68 &     2 &         &                &  & \\
352 & 1987 & 68 &     1 &         &                &  & \\
353 & 1986 & 67 &    12 &         &                &  & \\
354 & 1986 & 67 &    11 &         &                &  & \\
355 & 1986 & 67 &    10 &         &                &  & \\
356 & 1986 & 67 &     9 &         &                &  & \\
357 & 1986 & 67 &     8 &         &                &  & \\
358 & 1986 & 67 &   6/7 &         &                &  & \\
359 & 1986 & 67 &     5 &         &                &  & \\
360 & 1986 & 67 &     4 &         &                &  & \\
361 & 1986 & 67 &     3 &         &                &  & \\
362 & 1986 & 67 &     2 &         &                &  & \\
363 & 1986 & 67 &     1 &         &                &  & \\
364 & 1985 & 66 &    12 &         &                &  & \\
365 & 1985 & 66 &    11 &         &                &  & \\
366 & 1985 & 66 &    10 &         &                &  & \\
367 & 1985 & 66 &     9 &         &                &  & \\
368 & 1985 & 66 &     8 &         &                &  & \\
369 & 1985 & 66 &   6/7 &         &                &  & \\
370 & 1985 & 66 &     5 &         &                &  & \\
371 & 1985 & 66 &     4 &         &                &  & \\
372 & 1985 & 66 &     3 &         &                &  & \\
373 & 1985 & 66 &     2 &         &                &  & \\
374 & 1985 & 66 &     1 &         &                &  & \\
375 & 1984 & 65 &    12 &         &                &  & \\
376 & 1984 & 65 &    11 &         &                &  & \\
377 & 1984 & 65 &    10 &     282 & 1984-65-10-282 &  & \\
378 & 1984 & 65 &     9 &         &                &  & \\
379 & 1984 & 65 &     8 &         &                &  & \\
380 & 1984 & 65 &   6/7 &         &                &  & \\
381 & 1984 & 65 &     5 &         &                &  & \\
382 & 1984 & 65 &     4 &         &                &  & \\
383 & 1984 & 65 &     3 &         &                &  & \\
384 & 1984 & 65 &     2 &         &                &  & \\
385 & 1984 & 65 &     1 &         &                &  & \\
386 & 1983 & 64 &    12 &         &                &  & \\
387 & 1983 & 64 &    11 &         &                &  & \\
388 & 1983 & 64 &    10 &         &                &  & \\
389 & 1983 & 64 &     9 &         &                &  & \\
390 & 1983 & 64 &     8 &         &                &  & \\
391 & 1983 & 64 &   6/7 &         &                &  & \\
392 & 1983 & 64 &     5 &         &                &  & \\
393 & 1983 & 64 &     4 &         &                &  & \\
394 & 1983 & 64 &     3 &         &                &  & \\
395 & 1983 & 64 &     2 &         &                &  & \\
396 & 1983 & 64 &     1 &         &                &  & \\
397 & 1982 & 63 &    12 &         &                &  & \\
398 & 1982 & 63 &    11 &         &                &  & \\
399 & 1982 & 63 &    10 &         &                &  & \\
400 & 1982 & 63 &     9 &         &                &  & \\
401 & 1982 & 63 &     8 &         &                &  & \\
402 & 1982 & 63 &   6/7 &         &                &  & \\
403 & 1982 & 63 &     5 &         &                &  & \\
404 & 1982 & 63 &     4 &         &                &  & \\
405 & 1982 & 63 &     3 &         &                &  & \\
406 & 1982 & 63 &     2 &         &                &  & \\
407 & 1982 & 63 &     1 &         &                &  & \\
408 & 1982 & 63 &   6/7 &         &                & ??? sidetal er mærkelige & \\
409 & 1981 & 62 &     1 &      24 &   1981-62-1-24 &  & \\
410 & 1981 & 62 &     2 &   42-43 & 1981-62-2-42, 1981-62-2-43 &  & \\
411 & 1981 & 62 &     3 &   70-71 & 1981-62-3-70, 1981-62-3-71 & 62-3-a & \\
411.5 & 1981 & 62 &   3 &   70-71 & 1981-62-3-70, 1981-62-3-71 &  & Flere forsøg \\
412 & 1981 & 62 &     4 &     113 &  1981-62-4-113 &  & \\
413 & 1981 & 62 &     5 &     132 &  1981-62-5-132 &  & \\
414 & 1981 & 62 &   6/7 &     168 & 1981-62-67-168 &  & \\
415 & 1981 & 62 &     8 &     190 &  1981-62-8-190 & 62-8-aæ & \\
415 & 1981 & 62 &     8 &     190 &  1981-62-8-190 & 62-8-b & \\
416 & 1981 & 62 &     9 &     226 &  1981-62-9-226 &  62-9-a & Der er flere \\
416 & 1981 & 62 &     9 &     226 &  1981-62-9-226 &  62-9-b & Der er flere \\
416 & 1981 & 62 &     9 &     226 &  1981-62-9-226 &  62-9-c & Der er flere \\
417 & 1981 & 62 &    10 & 240-241 & 1981-62-10-240, 1981-62-10-241 &  & \\
418 & 1981 & 62 &    11 &     278 & 1981-62-11-278 &  & \\
419 & 1981 & 62 &    12 &     320 & 1981-62-12-320 &  & \\
420 & 1980 & 61 &     1 &      34 &   1980-61-1-34 & 61-1 & \\
421 & 1980 & 61 &     2 &      62 &   1980-61-2-62 & 61-2-a & \\
421 & 1980 & 61 &     2 &      62 &   1980-61-2-62 & 61-2-b & \\
422 & 1980 & 61 &     3 &      86 &   1980-61-3-86 &  & \\
423 & 1980 & 61 &     4 &      NA &    NA          &  & \\
424 & 1980 & 61 &     5 &     154 & 1980-61-5-154  &  & \\
425 & 1980 & 61 &   6/7 &     174 & 1980-61-67-173, 1980-61-67-174 &  & \\
426 & 1980 & 61 &     8 &    192f & 1980-61-8-192, 1980-61-8-193 &  & \\
427 & 1980 & 61 &     9 &    216f & 1980-61-9-216, 1980-61-9-217 &  & \\
428 & 1980 & 61 &    10 &     250 & 1980-61-10-250 &  & \\
429 & 1980 & 61 &    11 &     279 & 1980-61-11-279 &  & \\
430 & 1980 & 61 &    12 &     305 & 1980-61-12-305 &  & \\
431 & 1979 & 60 &     1 &      34 & 1979-60-1-34   & 60-1-a & \\
431 & 1979 & 60 &     1 &      34 & 1979-60-1-34   & 60-1-b & \\
432 & 1979 & 60 &     2 &      58 & 1979-60-2-58   & 60-2-a & \\
432 & 1979 & 60 &     2 &      58 & 1979-60-2-58   & 60-2-b & \\
432 & 1979 & 60 &     2 &      58 & 1979-60-2-58   & 60-2-c & \\
432 & 1979 & 60 &     2 &      58 & 1979-60-2-58   & 60-2-d & \\
433 & 1979 & 60 &     3 &      94 & 1979-60-3-94   & 60-3-a & \\
433 & 1979 & 60 &     3 &      94 & 1979-60-3-94   & 60-3-b & \\
433 & 1979 & 60 &     3 &      94 & 1979-60-3-94   & 60-3-c & \\
434 & 1979 & 60 &     4 &     118 & 1979-60-4-118  & 60-4 & \\
435 & 1979 & 60 &     5 &     154 & 1979-60-5-154  & 60-5-a & \\
435 & 1979 & 60 &     5 &     154 & 1979-60-5-154  & 60-5-b & \\
436 & 1979 & 60 &   6/7 &     186 & 1979-60-67-186 & 60-67-a & \\
436 & 1979 & 60 &   6/7 &     186 & 1979-60-67-186 & 60-67-b & \\
438 & 1979 & 60 &     8 &     214 & 1979-60-8-214  & 60-8-a & \\
438 & 1979 & 60 &     8 &     214 & 1979-60-8-214  & 60-8-b & \\
439 & 1979 & 60 &     9 &     254 & 1979-60-9-254  & 60-9-a & \\
439 & 1979 & 60 &     9 &     254 & 1979-60-9-254  & 60-9-b & \\
440 & 1979 & 60 &    10 &     286 & 1979-60-10-286 & 60-10-a & \\
440 & 1979 & 60 &    10 &     286 & 1979-60-10-286 & 60-10-b & \\
441 & 1979 & 60 &    11 &     322 & 1979-60-11-322 & 60-11-a & \\
441 & 1979 & 60 &    11 &     322 & 1979-60-11-322 & 60-11-b & \\
441 & 1979 & 60 &    11 &     322 & 1979-60-11-322 & 60-11-c & \\
441 & 1979 & 60 &    11 &     322 & 1979-60-11-322 & 60-11-d & \\
442 & 1979 & 60 &    12 &     354 & 1979-60-12-354 & 60-12-a & \\
442 & 1979 & 60 &    12 &     354 & 1979-60-12-354 & 60-12-b & \\
442 & 1979 & 60 &    12 &     354 & 1979-60-12-354 & 60-12-c & \\
442 & 1979 & 60 &    12 &     354 & 1979-60-12-354 & 60-12-d & \\
443 & 1978 & 59 &     1 &      38 & 1978-59-1-38   & 59-1-a & \\
443 & 1978 & 59 &     1 &      38 & 1978-59-1-38   & 59-1-b & \\
443 & 1978 & 59 &     1 &      38 & 1978-59-1-38   & 59-1-c & \\
443 & 1978 & 59 &     1 &      38 & 1978-59-1-38   & 59-1-d & \\
444 & 1978 & 59 &     2 &      74 & 1978-59-2-74   & 59-2 & \\
445 & 1978 & 59 &     3 &     114 & 1978-59-3-114  & 59-3-a & \\
445 & 1978 & 59 &     3 &     114 & 1978-59-3-114  & 59-3-b & \\
446 & 1978 & 59 &     4 &      NA &                &  & \\
447 & 1978 & 59 &     5 &      NA &                &  & \\
448 & 1978 & 59 &   6/7 &     238 & 1978-59-67-238 & 59-67-a & \\
448 & 1978 & 59 &   6/7 &     238 & 1978-59-67-238 & 59-67-b & \\
450 & 1978 & 59 &     8 &     278 & 1978-59-8-278  & 59-8-a & \\
450 & 1978 & 59 &     8 &     278 & 1978-59-8-278  & 59-8-b & \\
451 & 1978 & 59 &     9 &     322 & 1978-59-9-322  & 59-9  & \\
452 & 1978 & 59 &    10 &     358 & 1978-59-10-358 & 59-10-a & \\
452 & 1978 & 59 &    10 &     358 & 1978-59-10-358 & 59-10-b & \\
453 & 1978 & 59 &    11 &     394 & 1978-59-11-394 & 59-11 & \\
454 & 1978 & 59 &    12 &     430 & 1978-59-12-430 & 59-12-a & \\
454 & 1978 & 59 &    12 &     430 & 1978-59-12-430 & 59-12-b & \\
454 & 1978 & 59 &    12 &     430 & 1978-59-12-430 & 59-12-c & \\
455 & 1977 & 58 &     1 &         &                &  & Hele årgangen bestilt \\
456 & 1977 & 58 &     2 &         &                &  & \\
457 & 1977 & 58 &     3 &         &                &  & \\
458 & 1977 & 58 &     4 &         &                &  & \\
459 & 1977 & 58 &     5 &         &                &  & \\
460 & 1977 & 58 &     6 &         &                &  & \\
461 & 1977 & 58 &     7 &         &                &  & \\
462 & 1977 & 58 &     8 &         &                &  & \\
463 & 1977 & 58 &     9 &         &                &  & \\
464 & 1977 & 58 &    10 &         &                &  & \\
465 & 1977 & 58 &    11 &         &                &  & \\
466 & 1977 & 58 &    12 &         &                &  & \\
467 & 1976 & 57 &     1 &         &                &  & Hele årgangen bestilt\\
468 & 1976 & 57 &     2 &         &                &  & \\
469 & 1976 & 57 &     3 &         &                &  & \\
470 & 1976 & 57 &     4 &         &                &  & \\
471 & 1976 & 57 &     5 &         &                &  & \\
472 & 1976 & 57 &     6 &         &                &  & \\
473 & 1976 & 57 &     7 &         &                &  & \\
474 & 1976 & 57 &     8 &         &                &  & \\
475 & 1976 & 57 &     9 &         &                &  & \\
476 & 1976 & 57 &    10 &         &                &  & \\
477 & 1976 & 57 &    11 &         &                &  & \\
478 & 1976 & 57 &    12 &         &                &   & \\
479 & 1975 & 56 &     1 &         &                &  & \\
480 & 1975 & 56 &     2 &         &                &  & \\
481 & 1975 & 56 &     3 &         &                &  & \\
482 & 1975 & 56 &     4 &         &                &  & \\
483 & 1975 & 56 &     5 &         &                &  & \\
484 & 1975 & 56 &     6 &         &                &  & \\
485 & 1975 & 56 &     7 &         &                &  & \\
486 & 1975 & 56 &     8 &         &                &  & \\
487 & 1975 & 56 &     9 &         &                &  & \\
488 & 1975 & 56 &    10 &         &                &  & \\
489 & 1975 & 56 &    11 &         &                &  & \\
490 & 1975 & 56 &    12 &         &                &  & \\
491 & 1974 & 55 &     1 &         &                &  & \\
492 & 1974 & 55 &     2 &         &                &  & \\
493 & 1974 & 55 &     3 &         &                &  & \\
494 & 1974 & 55 &     4 &         &                &  & \\
495 & 1974 & 55 &     5 &         &                &  & \\
496 & 1974 & 55 &     6 &         &                &  & \\
497 & 1974 & 55 &     7 &         &                &  & \\
498 & 1974 & 55 &     8 &         &                &  & \\
499 & 1974 & 55 &     9 &         &                &  & \\
500 & 1974 & 55 &    10 &         &                &  & \\
501 & 1974 & 55 &    11 &         &                &  & \\
502 & 1974 & 55 &    12 &         &                &  & \\
503 & 1973 & 54 &     1 &         &                &  & \\
504 & 1973 & 54 &     2 &         &                &  & \\
505 & 1973 & 54 &     3 &         &                &  & \\
506 & 1973 & 54 &     4 &         &                &  & \\
507 & 1973 & 54 &     5 &         &                &  & \\
508 & 1973 & 54 &     6 &         &                &  & \\
509 & 1973 & 54 &     7 &         &                &  & \\
510 & 1973 & 54 &     8 &         &                &  & \\
511 & 1973 & 54 &     9 &         &                &  & \\
512 & 1973 & 54 &    10 &         &                &  & \\
513 & 1973 & 54 &    11 &         &                &  & \\
514 & 1973 & 54 &    12 &         &                &  & \\
515 & 1972 & 53 &     1 &         &                &  & \\
516 & 1972 & 53 &     2 &         &                &  & \\
517 & 1972 & 53 &     3 &         &                &  & \\
518 & 1972 & 53 &     4 &         &                &  & \\
519 & 1972 & 53 &     5 &         &                &  & \\
520 & 1972 & 53 &     6 &         &                &  & \\
521 & 1972 & 53 &     7 &         &                &  & \\
522 & 1972 & 53 &     8 &         &                &  & \\
523 & 1972 & 53 &     9 &         &                &  & \\
524 & 1972 & 53 &    10 &         &                &  & \\
525 & 1972 & 53 &    11 &         &                &  & \\
526 & 1972 & 53 &    12 &         &                &  & \\
527 & 1971 & 52 &     1 &         &                &  & \\
528 & 1971 & 52 &     2 &         &                &  & \\
529 & 1971 & 52 &     3 &         &                &  & \\
530 & 1971 & 52 &     4 &         &                &  & \\
531 & 1971 & 52 &     5 &         &                &  & \\
532 & 1971 & 52 &     6 &         &                &  & \\
533 & 1971 & 52 &     7 &         &                &  & \\
534 & 1971 & 52 &     8 &         &                &  & \\
535 & 1971 & 52 &     9 &         &                &  & \\
536 & 1971 & 52 &    10 &         &                &  & \\
537 & 1971 & 52 &    11 &         &                &  & \\
538 & 1971 & 52 &    12 &         &                &  & \\
539 & 1970 & 51 &     1 &         &                &  & \\
540 & 1970 & 51 &     2 &         &                &  & \\
541 & 1970 & 51 &     3 &         &                &  & \\
542 & 1970 & 51 &     4 &         &                &  & \\
543 & 1970 & 51 &     5 &         &                &  & \\
544 & 1970 & 51 &     6 &         &                &  & \\
545 & 1970 & 51 &     7 &         &                &  & \\
546 & 1970 & 51 &     8 &         &                &  & \\
547 & 1970 & 51 &     9 &         &                &  & \\
548 & 1970 & 51 &    10 &         &                &  & \\
549 & 1970 & 51 &    11 &         &                &  & \\
550 & 1970 & 51 &    12 &         &                &  & \\
551 & 1969 & 50 &     1 &         &                &  & \\
552 & 1969 & 50 &     2 &         &                &  & \\
553 & 1969 & 50 &     3 &         &                &  & \\
554 & 1969 & 50 &     4 &         &                &  & \\
555 & 1969 & 50 &     5 &         &                &  & \\
556 & 1969 & 50 &     6 &         &                &  & \\
557 & 1969 & 50 &     7 &         &                &  & \\
558 & 1969 & 50 &     8 &         &                &  & \\
559 & 1969 & 50 &     9 &         &                &  & \\
560 & 1969 & 50 &    10 &         &                &  & \\
561 & 1969 & 50 &    11 &         &                &  & \\
562 & 1969 & 50 &    12 &         &                &  & \\
563 & 1968 & 49 &     1 &         &                &  & \\
564 & 1968 & 49 &     2 &         &                &  & \\
565 & 1968 & 49 &     3 &         &                &  & \\
566 & 1968 & 49 &     4 &         &                &  & \\
567 & 1968 & 49 &     5 &         &                &  & \\
568 & 1968 & 49 &     6 &         &                &  & \\
569 & 1968 & 49 &     7 &         &                &  & \\
570 & 1968 & 49 &     8 &         &                &  & \\
571 & 1968 & 49 &     9 &         &                &  & \\
572 & 1968 & 49 &    10 &         &                &  & \\
573 & 1968 & 49 &    11 &         &                &  & \\
574 & 1968 & 49 &    12 &         &                &  & \\
575 & 1967 & 48 &     1 &         &                &  & \\
576 & 1967 & 48 &     2 &         &                &  & \\
577 & 1967 & 48 &     3 &         &                &  & \\
578 & 1967 & 48 &     4 &         &                &  & \\
579 & 1967 & 48 &     5 &         &                &  & \\
580 & 1967 & 48 &     6 &         &                &  & \\
581 & 1967 & 48 &     7 &         &                &  & \\
582 & 1967 & 48 &     8 &         &                &  & \\
583 & 1967 & 48 &     9 &         &                &  & \\
584 & 1967 & 48 &    10 &         &                &  & \\
585 & 1967 & 48 &    11 &         &                &  & \\
586 & 1967 & 48 &    12 &         &                &  & \\
587 & 1966 & 47 &     1 &         &                &  & \\
588 & 1966 & 47 &     2 &         &                &  & \\
589 & 1966 & 47 &     3 &         &                &  & \\
590 & 1966 & 47 &     4 &         &                &  & \\
591 & 1966 & 47 &     5 &         &                &  & \\
592 & 1966 & 47 &     6 &         &                &  & \\
593 & 1966 & 47 &     7 &         &                &  & \\
594 & 1966 & 47 &     8 &         &                &  & \\
595 & 1966 & 47 &     9 &         &                &  & \\
596 & 1966 & 47 &    10 &         &                &  & \\
597 & 1966 & 47 &    11 &         &                &  & \\
598 & 1966 & 47 &    12 &         &                &  & \\
599 & 1965 & 46 &     1 &      NA &                &  & \\
600 & 1965 & 46 &     2 &      NA &                &  & \\
601 & 1965 & 46 &     3 &      NA &                &  & \\
602 & 1965 & 46 &     4 &      NA &                &  & \\
603 & 1965 & 46 &     5 &      NA &                &  & \\
604 & 1965 & 46 &     6 &      NA &                &  & \\
605 & 1965 & 46 &     7 &      NA &                &  & \\
606 & 1965 & 46 &     8 &      NA &                &  & \\
607 & 1965 & 46 &     9 &      NA &                &  & \\
608 & 1965 & 46 &    10 &      NA &                &  & \\
609 & 1965 & 46 &    11 &      NA &                &  & \\
610 & 1965 & 46 &    12 &      NA &                &  & \\
611 & 1964 & 45 &     1 &         &                &  & \\
612 & 1964 & 45 &     2 &         &                &  & \\
613 & 1964 & 45 &     3 &         &                &  & \\
614 & 1964 & 45 &     4 &         &                &  & \\
615 & 1964 & 45 &     5 &         &                &  & \\
616 & 1964 & 45 &     6 &         &                &  & \\
617 & 1964 & 45 &     7 &         &                &  & \\
618 & 1964 & 45 &     8 &         &                &  & \\
619 & 1964 & 45 &     9 &         &                &  & \\
620 & 1964 & 45 &    10 &         &                &   & \\
621 & 1964 & 45 &    11 &         &                &  & \\
622 & 1964 & 45 &    12 &         &                &  & \\
623 & 1963 & 44 &     1 &      NA &                &  & \\
624 & 1963 & 44 &     2 &      NA &                &  & \\
625 & 1963 & 44 &     3 &      NA &                &  & \\
626 & 1963 & 44 &     4 &      NA &                &  & \\
627 & 1963 & 44 &     5 &      NA &                &  & \\
628 & 1963 & 44 &     6 &      NA &                &  & \\
629 & 1963 & 44 &     7 &      NA &                &  & \\
630 & 1963 & 44 &     8 &      NA &                &  & \\
631 & 1963 & 44 &     9 &      NA &                &  & \\
632 & 1963 & 44 &    10 &      NA &                &  & \\
633 & 1963 & 44 &    11 &      NA &                &  & \\
634 & 1963 & 44 &    12 &      NA &                &  & \\
635 & 1962 & 43 &     1 &      NA &                &  & \\
636 & 1962 & 43 &     2 &      NA &                &  & \\
637 & 1962 & 43 &     3 &      NA &                &  & \\
638 & 1962 & 43 &     4 &      NA &                &  & \\
639 & 1962 & 43 &     5 &      NA &                &  & \\
640 & 1962 & 43 &     6 &      NA &                &  & \\
641 & 1962 & 43 &     7 &      NA &                &   & \\
642 & 1962 & 43 &     8 &      NA &                &  & \\
643 & 1962 & 43 &     9 &      NA &                &  & \\
644 & 1962 & 43 &    10 &      NA &                &  & \\
645 & 1962 & 43 &    11 &      NA &                &  & \\
646 & 1962 & 43 &    12 &      NA &                &  & \\
\end{longtable}


\end{document}
