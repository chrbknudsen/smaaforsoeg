\emne{Emil Petersen – og et småforsøg}


\danskkemi{Dansk Kemi 2006, 87(5) p 46}



»Kobbersulfat prøves for Jern; 2 Gr. opløses i Vand, der tilsættes lidt Salpetersyre og Opløsningen inddampes. Derefter overmættes den med Ammoniak og fi ltreres; et Indhold af Jern vil
da vise sig som en rødbrun Rest af Jerntveiltehydrat paa Filtret«.
Ovenstående småforsøg er beskrevet af Emil Petersen og offentliggjort i hans Titreranalytiske Methoder, der udkom i 1905.
Emil Petersen blev født den 12. april 1856. Han blev døbt
Christian Emil Ulrich Petersen, men undlod altid mellemnavnene og kaldte sig Emil Petersen – så det vil vi også gøre her.
Emil Petersen blev optaget på Den Polytekniske Læreanstalt
og bestod kandidateksamen i Anvendt Naturvidenskab i 1879.
Han var derefter en kort tid ansat ved Elghammars Järnvärk i
Småland, hvor han udarbejdede en metode til fremstilling af
vanadiumpræparater af slaggen fra malm fra Taberg. Han rejste
tilbage til Danmark og bestod studentereksamen i 1881. I 1888
blev han dr.phil. på afhandlingen Vanadinet og dets nærmeste
Analoger.

I 1870-1872 var han frivillig lærling i marinen, 1872-1874
elev på søofficersskolen. 1881-1882 assistent for S.M. Jørgensen
ved Kemisk Laboratorium på Den Polytekniske Læreanstalt og
1882-1883 i Paris, hvor han studerede naturvidenskabshistorie.

Efter hjemkomsten skrev han en artikelserie i »Tilskueren« om
franske naturvidenskabsmænd. 1885-1893 assistent ved Universitetets kemiske laboratorium, 1895-1901 docent og 1901-1907
professor samme sted.

Hans helbred var ikke godt: Den 2. juli 1907 døde han 51 år
gammel – alt for ung. Af mange historikere omtales han som
ham, der indførte Den Fysiske Kemi i Danmark.
